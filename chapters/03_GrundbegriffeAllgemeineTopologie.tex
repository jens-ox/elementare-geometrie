\chapter{Grundbegriffe der allgemeinen Topologie}

\section{Toplogischer Räume}

\begin{definition}{Topologischer Raum}
  Ein \emph{topologischer Raum} ist ein Paar $ (X, \O) $ bestehend aus einer Menge $ X $ und einem System bzw. einer Familie 
  \begin{equation*}
    \O \subseteq \mathcal{P}(X)
  \end{equation*}
  von Teilmengen von $ X $, so dass gilt
  \begin{enumerate}
    \item $ X, \varnothing \in \O $ 
    \item Durchschnitte von \emph{endlich} vielen und Vereinigungen von \emph{beliebig} vielen Mengen aus $ \O $ sind wieder in $ \O $.
  \end{enumerate}
  Ein solches System $ \O $ heißt \emph{Topologie} von $ X $. Die Elemente von $ \O $ heißen \emph{offene Teilmengen} von $ X $. \\
  $ A \subset X $ heißt \emph{abgeschlossen}, falls das Komplement $ X \setminus A $ offen ist.
\end{definition}

\begin{example}{Extrembeispiele}
  \begin{enumerate}
    \item Menge $ X $, \ $ \O_\text{trivial} \coloneqq \{ X, \varnothing \} $ ist die \emph{triviale Topologie}.
    \item Menge $ X $, \ $ \O_\text{diskret} \coloneqq \mathcal{P}(X) $ ist die \emph{diskrete Topologie}.
  \end{enumerate}
\end{example}

\begin{example}{Standard-Topologie auf $ \R $}
  \begin{marginfigure}[4em]
    \textbf{Offenes Intervall}: \\ $ (a,b) \coloneqq \{  t \in \R : a < t < b \} $, \\ $ a $ und $ b $ beliebig
  \end{marginfigure}
  $ X = \R $,
  \begin{equation*}
    \O_{s \text{ (standard)}} \coloneqq \{ I \subset \R : I = \text{Vereinigung von offenen Intervallen} \}
  \end{equation*}
  ist Topologie auf $ \R $.
\end{example}

\clearpage

\begin{example}{Zariski-Topologie auf $ \R $}
  $ X = \R $,
  \begin{equation*}
    \O_\text{Z(ariski)} \coloneqq \{ O \subset \R : O = \R \setminus, \ E \subset \R \text{ endlich} \} \cup \{ \varnothing \}
  \end{equation*}
  ist die \emph{Zariski-Topologie} auf $ \R $. \\
  (Mit anderen Worten: Die abgeschlossenen Mengen sind genau die endlichen Mengen, $ \varnothing $ und $ \R $.) \\
  Diese Topologie spielt eine wichtige Rolle in der algebraischen Geometrie beim Betrachten von Nullstellen von Polynomen: \\
  \begin{align*}
    (a_1 \dots, a_n) &\leftrightarrow p(X) = (X-a_1)\cdots(X-a_n) \\
     \R &\leftrightarrow \text{ Nullpolynom} \\
     \varnothing &\leftrightarrow X^2+1
  \end{align*}
  % TODO Abbildung 2 einfügen
\end{example}

\begin{definition}{Metrischer $ \to $ topologischer Raum}
  Metrische Räume (z.B. $ (X, d) $) sind topologische Räume: \\
  % TODO Abbildung 3
  $ U \subset X $ ist \emph{$ d $-offen} $ \Leftrightarrow \forall p \in U \ \exists \ \epsilon = \epsilon(p) > 0 $, sodass der offene Ball $ B_\epsilon(p) = \{ x \in X : d(x,p) < \epsilon \} $ um $ p $ mit Radius $ \epsilon $ ganz in $ U $ liegt: $ B_\epsilon(p) \subset U $. \\
  Die $ d $-offenen Mengen bilden eine Topologie --- die von der Metrik $ d $ \emph{induzierte Topologie}\sidenote{\textbf{Übungsaufgabe}: Zeigen, dass es sich wirklich um eine Topologie handelt}.
\end{definition}

\begin{definition}{Basis}
  Eine \emph{Basis} für die Topologie $ \O $ ist eine Teilmenge $ \mathcal{B} \subset \O $, sodass für jede offene Menge $ \varnothing \neq V \in \O $ gilt:
  \begin{equation*}
    V = \bigcup_{i \in I}V_i, \quad V_i \in \mathcal{B}\text{.}
  \end{equation*}
  \underline{Beispiel}: $ \mathcal{B} = \{ \text{offene Intervalle} \} $ für Standard-Topologie auf $ \R $.
\end{definition}

\begin{example}{Komplexität einer Topologie}
  $ \R $, $ \C $ haben eine abzählbare Basis bezüglich Standard-Metrik $ d(x,y) = \vert x - y \vert $ (beziehungsweise Standard-Topologie): \\
  Bälle mit rationalen Radien und rationalen Zentren.
  % TODO Abbildung 4
\end{example}

\begin{bla}{Bemerkung --- Gleichheit von Topologien}
  Verschiedene Metriken können die gleiche Topologie induzieren: \\
  Sind $ d, d' $ Metriken auf $ X $ und enthält jeder Ball um $ x \in X $ bezüglich $ d $ einen Ball um $ x $ bezüglich $ d' $ ($ B_{\epsilon'}^d(x) \subset B_\epsilon^d(x) $), dann ist jede $ d $-offene Menge auch $ d' $-offen und somit $ \O(d) \subset \O(d') $. \\
  Gilt auch die Umkehrung ($ \O(d') \subset \O(d) $), so sind die Topologien gleich: $ \O(d) = \O(d') $.
\end{bla}

\begin{example}{Bälle und Würfel sind gleich}
  $ X = \R^2 $, $ x = (x_1, x_2) $, $ y = (y_1, y_2) $
  \begin{align*}
    d(x,y) &\coloneqq \sqrt{(x_1-y_1)^2+(x_2-y_2)^2} \\
    d'(x,y) &\coloneqq \max\{ \vert x_1-y_1\vert, \ \vert x_2-y_2 \vert \}
  \end{align*}
  Die induzierten Topologien sind gleich.
  % TODO Abbildung 5
\end{example}

\begin{example}{Metrische Information sagt nichts über Topologie}
  $ (X, d) $ sei ein beliebiger metrischer Raum,
  \begin{equation*}
    d'(x,y) \coloneqq \frac{d(x,y)}{1 + d(x,y)}
  \end{equation*}
  ist Metrik mit $ \O(d) = \O(d') $. \\
  Für $ d' $ gilt: $ d'(x,y) \leq $ ($ \forall x,y $), insbesondere ist der Durchmesser von $ X $ bezüglich $ d' $:
  \begin{equation*}
    = \sup_{x,y \in X}d'(x,y) \leq 1\text{,}
  \end{equation*}
  das heißt, der Durchmesser eines metrischen Raumes (``metrische Information'') sagt nichts über die Topologie aus.
\end{example}

\begin{definition}{Umgebung}
  % TODO Abbildung 6
  $ (X, \O) $ sei ein topologischer Raum. $ U \subset X $ heißt \emph{Umgebung} von $ A \subset X $, falls
  \begin{equation*}
    \exists \ O \in \O : A \subset O \subset U\text{.}
  \end{equation*}
\end{definition}

\begin{definition}{Innerer Punkt}
  Für $ A \subset X $, $ p \in X $ heißt $ p $ ein \emph{innerer Punkt} von $ A $ (bzw. äußerer Punkt von $ A $), falls $ A $ (bzw. $ X \setminus A $) Umgebung von $ \{ p \} $ ist. \\
  Das \emph{Innere} von $ A $ ist die Menge $ \overset{\circ}{A} $ der inneren Punkte von $ A $.
\end{definition}

\begin{definition}{Abgeschlossene Hülle}
  Die \emph{abgeschlossene Hülle} von $ A $ ist die Menge $ \overline{A} \subset X $, die \underline{nicht} äußere Punkte sind. \\
  \textbf{Beispiel}: $ (a, b) = \{ t \in \R : a < t < b \} $, \\ $ \overline{(a,b)} = [a,b] = \{ t \in \R : a \leq t \leq b \} $.
\end{definition}

\begin{bla}{Drei konstruierte topologische Räume}
  Folgende drei einfache Konstruktionen von neuen topologischen Räumen aus gegebenen:
  \begin{enumerate}

    % -- 1
    \item \textbf{Teilraum-Topologie}: $ (X, \O_X) $ topologischer Raum, $ Y \subseteq X $ Teilmenge.
    \begin{equation*}
      \O_Y \coloneqq \{ U \subseteq Y : \exists \ V \in \O_X \wedge U = V \cap Y \}
    \end{equation*}
    definiert eine Topologie auf $ Y $, die sogenannte \emph{Teilraum-Topologie}.\sidenote{Zu überprüfen!} \\
    \textbf{Achtung!} $ U \in \O_Y $ ist i.a. \underline{nicht} offen in $ X $. Z.B. $ X = \R $, $ Y = [0,1] $, $ V = (-1, 2) $, also $ U = V \cap Y = Y $.
    % TODO Abbildung 1

    % -- 2
    \item \textbf{Produkträume}: $ (X, \O_X) $ und $ (Y, \O_Y) $ zwei topologische Räume. Eine Teilmenge $ W \subseteq X \times Y $ ist \emph{offen} in der \emph{Produkt-Topologie} $ \Leftrightarrow \ \forall (x, y) \in W \ \exists $ Umgebung $ U $ von $ x $ in $ X $ und $ V $ von $ y $ in $ Y $ sodass das ``Kästchen'' $ U \times V \subseteq W $. \\
    \textbf{Achtung!} Nicht jede offene Menge in $ X \times Y $ ist ein Kästchen: die Vereinigung von zwei Kästchen ist beispielsweise auch offen. \\
    \textbf{Beispiel}: $ X = \R $ mit Standard-Topologie, dann ist
    \begin{equation*}
      \underbrace{X \times \cdots \times X}_{x \text{ mal}} = \R^n
    \end{equation*}
    induzierter topologischer Raum.
    % TODO Abbildung 2

    % -- 3
    \item \textbf{Quotienten}: $ (X, \O) $ topologischer Raum, $ \sim $ Äquivalenzrelation\sidenote{Impliziert Partitionierung von $ X $ in disjunkte Teilmengen} auf $ X $. Für $ x \in X $ sei
    \begin{equation*}
      [x] \coloneqq  \{ y \in X : y \sim x \}
    \end{equation*}
    die Äquivalenzklasse von $ x $,
    \begin{equation*}
      X/\sim
    \end{equation*}
    die Menge der Äquivalenzklassen und
    \begin{align*}
      \pi : X &\to X/\sim \\
      x &\mapsto [x]
    \end{align*}
    die kanonische Projektion (surjektiv!). \\
    Die \emph{Quotienten-Topologie} auf $ X/\sim $ nutzt: \\
    $ U \subset X/\sim $ ist \underline{offen} $ \overset{\text{Def.}}{\Leftrightarrow} \pi^{-1}(U) $ ist offen in $ X $.
  \end{enumerate}
\end{bla}