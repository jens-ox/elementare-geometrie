\chapter{Spezielle Klassen von topologischen Räumen}

\begin{bla}{Übersicht}
  Folgende spezielle Klassen sollen diskutiert werden:
  \begin{itemize}
    \item metrische Räume $ \leadsto $ metrische Geometrie
    \item Mannigfaltigkeiten (Grundobjekte in Differenzialgeometrie, Physik,\dots)
    \item Polyeder, Simplizialkomplexe (Kombinatorik, algebraische Topologie)
    \item Bahnen-Räume von Gruppenaktionen (geometrische Gruppentheorie)
  \end{itemize}
\end{bla}

\section{Topologische Mannigfaltigkeiten}

\begin{definition}{Topologische Mannigfaltigkeit}
  Eine topologische Mannigfaltigkeit ist ein topologischer Raum $ M $ mit Folgenden Eigenschaften:
  \begin{enumerate}
    \item $ M $ ist \emph{lokal euklidisch}, d.h. $ \forall p \in M \ \exists $ offene Umgebung $ U $ von $ p $ und ein Homöomorphismus $ \varphi: U \to \varphi(U) \subset \R^n $ mit festem $ n $. Das Paar $ (\varphi, U) $ heißt \emph{Karte}\sidenote{Eine mathematische Karte ist einer echten Karte ähnlich. Man nehme einen Punkt, zum Beispiel Karlsruhe, und beschreibt die Umgebung von Karlsruhe in Form einer Karte auf einer DIN A4-Karte. Das ist natürlich nicht bijektiv, aber man versucht es möglichst bijektiv zu machen.} und eine Menge $ \mathcal{A} = \left\{ (\varphi_a, U_\alpha) : \alpha \in A \right\} $ mit $ \bigcup_{\alpha \in A}U_\alpha = M $ heißt \emph{Atlas}.
  \end{enumerate}
\end{definition}