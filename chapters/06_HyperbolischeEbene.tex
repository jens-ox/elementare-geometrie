\chapter{Nichteuklidische Geometrie --- Hyperbolische Ebene}

Die euklidische Geometrie verfolgt einen axiomatischen Zugang --- es ist beispielsweise nicht näher definiert, was ein Punkt ist. Genauso gibt es das Parallelen-Axiom, welches besagt, dass es zu einer gegebenen Gerade \( g \) und einem Punkt \( P \), der nicht auf dieser Geraden liegt, genau eine Gerade gibt, die parallel zu \( g \) ist und \( P \) beinhaltet. Es wurde lange versucht, das Parallelen-Axiom aus anderen Axiomen zu konstruieren, allerdings gelang das nicht. \\
Um 1900 wurde von Poincaré und Klein die hyperbolische Ebene formalisiert.

\begin{definition}[Hyperbolische Ebene]
  Es sei \( H^2 \coloneqq \left \{ \left( x_1, x_2 \right) \in \R^2 : x_2 > 0 \right \} \) die obere Halbebene. Es seien
  \begin{itemize}
    \item \term{Punkte} die Elemente in \( H^2 \) und
    \item \term{Geraden} die Halbkreise mit Zentrum auf der \( x_1 \)-Achse und die Parallelen zur \( x_2 \)-Achse.
  \end{itemize}
  Leicht lässt sich zeigen, dass es wie in der euklidischen Geometrie auf der hyperbolischen Ebene eine Gerade zwischen zwei beliebiegen Punkten gibt. Allerdings ist diese Gerade hier im Allgemeinen nicht eindeutig. Das Parallelen-Axiom gilt auf der hyperbolischen Ebene nicht, da hier zu gegebener Gerade \( g \) und Punkt \( P \) mehrere Geraden \( \widetilde{g_1}, \widetilde{g_2}, \dots \) gefunden werden können, sodass
  \begin{equation*}
    \widetilde{g_1} \cap g = \widetilde{g_2} \cap g = \cdots = \varnothing\text{.}
  \end{equation*}
\end{definition}